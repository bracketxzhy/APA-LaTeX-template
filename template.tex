\documentclass{scupi_apa_thesis}

\title{A Review of Factors Correlated With School Alienation Among College Students}
\author{Student Name}
\studentid{2023123456}
\instructor{Professor Yumei Li}
\course{Final Paper for ENGCMP200-Section 12}
\institute{Sichuan University-Pittsburgh Institute}
\date{\today}

\begin{document}
\maketitle
\par
Alienation, first proposed by the famous German philosopher, Georg Wilhelm Friedrich 
Hegel, refers to the process of one’s spirit transferring from an object to itself, which means that 
the spirit is separated from the body (Hegel, 1807/2018). Over time, human alienation mainly 
manifests in powerlessness, meaninglessness, and normlessness, respectively representing people 
feeling helpless to control events, failing to see the prospects, and losing sense of rules (Brown et 
al., 2003). In recent years, school alienation (SAL) is frequently mentioned in the educational 
environment and manifests in three domains. Firstly, alienated students might feel disconnected 
from the learning processes at school; secondly, they might feel distanced from their teachers; 
and thirdly, they may feel isolated from their classmates (Morinaj \& Hascher, 2018). Research 
shows that many college students have been facing school alienation recently, failing to see the 
value of schooling, lacking a sense of well-being, getting clinical depression, or even committing 
suicide (Zhang, 2003). As the negative impacts of school alienation are quite serious, it is 
imperative for us to identify the primary elements that are correlated with school alienation 
among university students. To meet this urgency, this literature review will examine current 
articles on this topic, investigating the factors related to college students’ school alienation in 
recent years. The research question of this paper is: what are the factors correlated with school 
alienation among college students? 
\section{method}
A search of the EBSCO-Education Research Information Center (ERIC) and 
ScienceDirect electronic databases was undertaken by using the search terms “college students” 
or “Chinese college students” combined with “alienation”, “school alienation”, “student 
alienation”, “campus life”, “mental health”, “self-evaluation” and “dissimilation” in various 
ways. After the initial search, only studies published within the years of 2010-2024 were 
selected. Studies that review the literature were excluded. After implementing the inclusion and 
exclusion criteria and removing duplicates, the paper selected six studies for analysis (see Table 
1 for included studies).
\par
The research question is: what are the factors correlated with school alienation among 
college students? Based on the findings of these studies, I have identified three key themes in 
terms of personal and external factors that may answer the research question: students’ self
perception, school learning culture, and education system.

\section{findings}
\subsection{Students’ Self-Perception }
\par
Studies have indicated that there is a correlation between self-perception and school 
alienation among college students. To understand the concept of self-perception, Caselman \& 
Self (2007) have illustrated that self-perception is the understanding of oneself that is built from 
the interaction of the person and his or her daily life experiences. A study by David \& Nită 
(2014) came to the conclusion that self-perception has a high negative correlation to alienation. 
They analyzed 435 questionnaires to the freshmen with a mean age of 20.87 from two study 
programs in Romania, utilizing instruments like the Personal Evaluation Inventory (PEI), 
Mastery Scale, Alienation Scale, Revised Philosophies of Human Nature Scale, and Faith in 
People.  After using linear regressions, the result showed that there existed a strong and negative 
correlation between self-perception and school alienation (r=-0.66), which indicated that if the 
freshmen had negative self-perception, then the person was likely to experience school 
alienation.
\par
Similarly, Wu et al. (2024) ’s findings supported this correlation. They distributed 2932 
web-based questionnaires to undergraduate students ranging from 17 to 24 years old in nine 
universities in China. The scales used in the questionnaires came from the Adolescent Students’ 
Alienation Scale (ASAS), Adolescent Students Life Satisfaction Scale (ASLSS), and Security 
Questionnaire (SQ). After that, they applied SPSS 25.0 software to analyze data. 
After regression analysis, the result also provided strong evidence that people who believed they had low self
worth and low life satisfaction levels experienced school alienation. 
\par
To sum up, the lower the level of self-perception, the higher the level of alienation a 
student will have (David \& Nită, 2014). Both studies strongly demonstrate the significant role of 
self-perception considering the correlation with school alienation among college students.

\subsection{School Learning Culture}
\par
Besides the factor of students’ negative self-perception, research has shown that 
the school learning culture also has a link with school alienation for college students. A study by 
Caglar (2013) indicated that students’ perception of the fairness of the school learning 
environment is a significant predictor of their feelings of alienation. Specifically, the concept of 
fairness or justice in the field of educational organizations, is defined as “conformity to what is 
right and legal, conceptualized into three aspects: distributive justice, procedural justice, and 
interactional justice” (Caglar, 2013, p. 185). In this study, 2600 undergraduate students studying 
at Adiyaman University School of Education participated and filled in Fair Learning 
Environment Questionnaire. After using a stratified sampling technique and excluding 
incomplete or defective forms, the study sample included 952 forms for further analysis. After 
applying normally distributed independent samples t-test and a multiple linear regression 
analysis, the research found a negative level of significant correlation (r=-0.40) between the fair 
school learning environment and school alienation. The result shows that the less students 
perceive their learning environment to be fair, the more they experience dissatisfaction and 
alienation.
\par
Additionally, Trivedi \& Prakasha (2021) highlighted that there also existed a strong 
inverse correlation between students’ school alienation and the school’s organizational culture. 
For higher education institutions, organizational culture can be explained with two aspects: the 
form and extent of control, and the emphasis on policy and strategy (van der Velden, 2012). In 
order to find out whether students’ school alienation and organizational culture are correlated, the 
study selected a sample of 600 undergraduate students in one of the top universities in India. 
With the method of descriptive survey research, the study used a student alienation scale and 
organizational culture assessment instrument to collect data. A Pearson correlation test and a 
regression statistical test were also conducted. Finally, a moderate negative correlation was 
observed between student alienation and the culture of the school organization (r=-0.448), and 
29.1\% of the alienation among college students was explained by the organizational culture on 
the campus.
\par
In summary, the learning culture in a university is correlated with school alienation 
among college students. Students’ experience and engagement are partly influenced, indicating 
that they are less alienated if the school has a fair and better learning culture (Trivedi \& 
Prakasha, 2021). 

\subsection{Education System}
\par
In addition, literature has also found the education system to be another relevant factor 
associated with students’ school alienation. Starting with Marx's theory of labor alienation, Xia 
(2024) analyzed the manifestations of educational alienation and deduced that the defects of the 
current education system were the external cause of students’ alienation. The author highlighted 
that the deficiencies in the education system were mainly manifested in the imbalance of 
educational resources, the use of good and bad grades as the criteria for evaluating learning 
outcomes, and the use of academic qualifications as a condition for admission to organizations. 
Under the influence of these factors, college students were bounded by control of grades and 
diplomas, facing school alienation. This phenomenon shows that the education system is highly 
correlated with students’ school alienation.
\par
Similarly, a recent analysis by Deng (2024) has supported the assertion that the education 
system is a factor correlated with school alienation among college students. The study conducted 
semi-structured interviews involving 20 college students from across China, including Chongqing, 
Shandong, Beijing, and Ningbo. During this process, the interview lasted approximately one hour 
and included the keywords like “exam”, “teaching methods”, “education system” and so on. After 
the interviews, the study analyzed the qualitative data using a grounded theory approach to 
conceptualize and generalize the data. Finally, the author concluded that an education system with 
test-oriented education is related to college students’ school alienation. He also claimed that “the 
education system ceases to nurture critically thinking individuals and instead operates as an 
assembly line for training players who can master the rules of the examination game” (Deng, 2024, 
p. 9).
\par
To sum up, the education system is a relevant factor associated with students’ school 
alienation, as it often prioritizes test scores over critical thinking and personal development, 
leading to a sense of disconnection among college students.
\section{conclusion}
\par
To answer what factors are correlated with school alienation among college students, this 
article reviewed six papers and found that school alienation is correlated with students’ self
perception, the school learning culture, and the education system. Specifically, those studies 
indicated that if college students had negative self-perceptions, experienced bad school learning 
culture or harsh education system, they were likely to experience school alienation. These studies 
used various research methods like semi-structured interviews, questionnaires, and thematic 
analysis, having important implications for understanding the correlation associated with school 
alienation and its potential impact on education and society.
\par
However, there are still some limitations in the existing literature, including geographical 
constraints and a lack of long-term follow-up studies. In addition, most studies fail to adequately 
consider the diversity and complexity of school alienation in different cultural contexts. 
Therefore, to research the factors correlated with school alienation among college students, 
future studies could study across a wider range of geographical places to capture the global 
diversity in educational and cultural contexts. Moreover, future research is advised to conduct 
long-term follow-up studies to gain a more objective overview and a more reliable conclusion.
\cite{CCTV2025}
\cite{Chen2021}
\cite{Eikeland2022}
\newpage

\bibliographystyle{apacite}
\bibliography{references}

\end{document}
