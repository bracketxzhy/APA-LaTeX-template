\documentclass{studentpaper}

\title{Should Universities Expand Enrollment?}
\author{Stefan Zhang}
\studentid{2024141520349}
\instructor{Professor Yumei Li}
\course{Final Paper for ENGCMP200-Section 12 }
\institute{Sichuan University-Pittsburgh Institute }
\date{\today}

\begin{document}
\maketitle
\par
On the third session of the 14th National People's Congress, the director of National Development and Reform Commission said China will continue to expand enrollment of “double world-class project” colleges. 
In 2025, universities plan to increase nearly 20,000 enrollments (CCTV, 2025). 
This announcement gained a wide range of social attention, which aroused boisterous argument on this published policy. 
Proponents argued that this policy will ease the pressure of senior students and raise the social education level. Critics, however, claimed that this will be useless for social development (Yao, 2014). 
Universities should not expand enrollment because it lowers educational quality, intensifies the job market, restricts talent diversity and reaches diminishing marginal utility of enrollment expansion.
\par
Expanding enrollment will lower the quality of college education due to the increasing student-faculty ratio. 
According to Yao (2023), universities in China have higher student-faculty ratio at 17.51:1, compared to some famous top universities from developed countries according to Times Higher Education (2016), whose ratios are generally lower. 
Students from lower student-faculty ratio universities can benefit from more personalized education, receiving more attention, tailored guidance and support from teachers. For instance, in Sichuan University Pittsburgh Institute, students have ENGCMP 140 course that has a maximum of only 7 students per class, which provide students a valuable opportunity to practice oral English. 
In contrast, an increase in student numbers may raise teaching pressure on faculty, making it difficult for them to offer individualized instructions. 
Some opponents may argue that universities can also hire more faculties to reach the requirement. 
However, in some field, high qualified faculty, as a scarce resource, are hard to expand compared to the expansion of student enrollment (Jerde, 2014). 
Faculty growth structurally lags behind enrollment surges. Consequently, as enrollment expands, the quality of education may decline because students receive less personalized support, hindering their academic growth and overall learning experience.
\par
The policy of expanding enrollment will intensify the job recruitment market, creating significant challenges for graduates seeking employment. 
There is a clear contradiction between the limited recruitment market and the growing number of graduates. 
According to The State Information Center (Chen, 2021), the unemployment rate for recent college graduates in China is higher than national-wide rate, representing significant employment pressure for students. 
Data from the Ministry of Education (Yao, 2023) revealed that the number of university students rose steadily from 2018 to 2023, indicating that an increasing number of graduates will enter the job market in the coming years. 
If enrollment continues to expand, it may further lead to higher graduate underemployment rate and dissatisfaction, gradually diminishing the value of college degree. Without targeted expansion that reflects labor market demands, the policy could overfill the market with graduates whose skills do not match available opportunities, deepening labor market inefficiencies.
\par
Excessively high enrollment rate enlarges the shortage of skilled professionals and restricts the diversity of talent cultivation. 
Therefore, it fails to meet society's demand for diverse talents. Nowadays, society needs not just scientists, but engineers, athletes and other skilled professionals. 
In fact, we are currently facing a shortage of high-skilled professionals in some specific fields as well as a lack of interdisciplinary talents (Hao, 2024). 
If we continue to expand enrollment, the shortage of such professionals is likely to become more severe.
\par
However, some critics question the link between the shortage of skilled professionals and the expansion of enrollment. 
They argue that different majors have different cultivation plans (Eikeland \& Ohna, 2022) which do not interfere with each other. 
For instance, students who pursue vocational education may embark on entirely different career paths after junior high school compared to those entering university track. 
Nevertheless, the correlation becomes clearer when we consider that the overemphasis on academic degrees has diverted both resources and students away from vocational education paths. 
According to Zhang (2024), the allocation priority of educational resources toward universities has led to significant shortages in vocational education, including in teaching staff and facilities support. 
Consequently, the resource imbalance weakens vocational schools’ ability to train skilled professionals. 
Many students with interests or special talents in some technical areas are compelled to follow the social trends of entering university because of the low quality of vocational education (Wang, 2023). 
For these people, expansion of university covers up their potential specialties. 
What they truly require is high-qualify vocational education resources and improvement in the occupational status rather than the continued expansion of university enrollment. 
Therefore, overhigh university enrollment rates enlarge the shortage of skilled professionals and restrict the diversity of talent development.
\par
Although the university enrollment expansion policy published in 1999 successfully improved access to higher education, its continued, undifferentiated expansion today may be reaching a point of diminishing marginal utility (DMU). 
This economics concept used here refers to additional net gain to society, considering both benefits and costs from enrolling more students. 
The expansion policy reaches DMU means the benefits ranging from expanding the pool of intellectual worker quantity to elevating the educational level of society gained from enrolling even more students are gradually diminishing compared to the costs of expansion. 
Currently, the job market may require more high-skilled top talents rather than a large number of employees with ordinary skills (Zhang et al., 2023). 
If expansion persists primarily focused on quantitative growth without fundamental structural reforms and a substantial shift towards cultivating advanced skills, the net social return on further enrollment increases is likely to diminish. 
This is because the costs associated with potential graduate underemployment, wasted educational investment, diluted and misallocated resources may increasingly surpass the benefits of simply having more degree holders, especially when their skills fail to match current job market demands.
\par
Therefore, while expanding university enrollment may seem beneficial on the surface, a deeper analysis reveals that it injures educational quality, unbalances labor market dynamics, and ultimately fails to serve the diverse and evolving needs of society.

\bibliographystyle{apacite}
\bibliography{references}

\end{document}
